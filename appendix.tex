\appendix
\onecolumn
\section{Proof of Competitive Ratio for \algoname Algorithm}
\label{appendix:virtual_plus_proof_k_2}

We now prove Theorem \ref{thm:K_2_theorem1} from the main paper reproduced here for convenience. 

\begin{reptheorem}{thm:K_2_theorem1}  For $k=2$, the competitive ratio achieved by \algoname\ algorithm is equal to,
\begin{equation}
    C_n = \frac{t(t-1)}{n} \sum_{j=t}^{n-1}\frac{1}{j(j-1)} \left(1 + 2 \sum_{ p=t+1}^j \frac{1}{p-1}\right)
\end{equation}
Particularly for $t= \alpha \cdot n\,,\, \alpha \in (0,1)$ we get 
\begin{equation} \label{eq:lower_bound_alpha1}
    C_n > \alpha ( 3(1-\alpha) + 2 \alpha \ln(\alpha)) + \mathcal{O}(1/n)
\end{equation}
Thus, asymptotically  we have
\begin{equation}
    C >  \max_{\alpha \in [0,1]} \alpha ( 3(1-\alpha) + 2 \alpha \ln(\alpha)) > .4273 > 1/e \, .
\end{equation}
\end{reptheorem}

\begin{figure}[ht]
    \centering
    \includegraphics[width=0.50\linewidth]{Figures/virtual_plus_k_2.pdf}
    %\includegraphics[width=\linewidth]{Figures/Competitive_RatioVar5.png}
    \caption{Probability of having only one element in $S_{\mathcal{A}}$ after $j$ time-steps with the \algoname\ algorithm.}
    \label{fig:k_2}
    %\vspace{-5mm}
\end{figure}

\begin{proof}
First note that we can show that the competitive ratio for the $k$-secretary problem is equal to 
\begin{equation}
    C = \frac{1}{k}\sum_{a=1}^k \mathbb{P}(i_a \in S_\mathcal{A}), \label{eq:C_as_sum_prob1}
\end{equation}
where $i_a$ is the index of the $a^{th}$ secretary picked by the offline solution ---i.e. $i_a$ is a top-$k$ secretary of $\mathcal{D}$. 
Now, let us focus on the case $k=2$.
When calculating the probability of either of the top two items in $\mathcal{D}$ being picked by the \algoname\ we must first compute the probability of one of the top-2 items being picked during the selection phase (time step $t+1 \dots n$). Now notice that \algoname\ picks an item at time step $j+1$ if and only if this is a top-2 item with respect to all of $\mathcal{D}$ and $|S_{\mathcal{A}}| \leq 2$ at time-step $j+1$. Let top-$2_j$ denote the two largest elements observed by $\mathcal{A}$ up to and inclusive of time step $j$. Thus, for $a \in \{1,2\}$, we have
\begin{align}
    \mathbb{P}(i_a \in S_\mathcal{A}) 
    &= \sum_{j=t}^{n-1} \mathbb{P}(i_a \in S_\mathcal{A} \text{ at time-step }j+1) \label{p_picked_equal_not_filled} \\
    &= \frac{1}{n}\sum_{j=t}^{n-1} \mathbb{P}(|S_{\mathcal{A}}| \leq 2 \text{ at time-step } j + 1) \notag
\end{align}

Now, we compute $\mathbb{P}(|S_{\mathcal{A}}| \leq 2 \text{ at time-step } j+1)$ by decomposing this probability into the following two events: A.) $|S_{\mathcal{A}}| = 0$ where the selection set is empty and B.) the event $|S_{\mathcal{A}}| = 1$ where exactly one item has been picked. We now analyze each event in turn.

\xhdr{Event A} In order for the event $|S_{\mathcal{A}}|=0$ to occur it implies that the algorithm does not select any items in the first $j$ rounds. This means both two top-$2_j$ elements must have appeared in the sampling phase. Thus the probability for this event is exactly $\frac{t(t - 1)}{j (j - 1)}$. 

\xhdr{Event B}
The second event is when $|S_{\mathcal{A}}|=1$ ---i.e. the algorithm picks exactly one element in the first $j$ rounds. The computation of this event is illustrated in Figure~\ref{fig:k_2}. Let's say that an element is picked at time step $p$. Now to compute the probability of Event B occurring we first make the following two observations:

\begin{enumerate}[label={\bf Observation \arabic*:}, topsep=0pt, parsep=0pt, leftmargin=69pt, itemsep=2pt]
    \item  In order for exactly one element to be picked at the time step $p \leq j$, this element must be one of the top-$2_j$ elements. Furthermore, this implies the other of the top-$2_j$ element ---i.e. the one not picked at $p$ must have appeared in the sampling phase. Note that if both top-$2_j$ elements appear after the sampling phase, the condition would be satisfied twice and two elements would be selected instead of exactly one, and if they both appeared during the sampling phase we return to Event A. As a result, the probability for this condition is given by $\frac{t}{j (j - 1)}$.  
    \item By observation 1. we know that the online algorithm $\mathcal{A}$ picks one of the top-$2_j$ at time step $p$ and the fact that the event under consideration is $|S_{\mathcal{A}}|=1$ the reference list $R$ from time step $p$ to $j+1$ must contain both top-$2_j$ elements. However, for $\mathcal{A}$ to pick \emph{only} at $p$ we also need to ensure that no elements are picked prior t to $p$. Therefore, before time step $p$ the reference list must contain  top-$2_{p}$ . Again by observation 1, we know that $R$ already contains one of the top-$2_j$ elements therefore we know it contains one of the  top-$2_p$ elements. Thus the probability of ensuring that the second  top-$2_p$ elements is also within $R$ by time step $p$ is $\frac{(t - 1)}{(p - 2)}$. Finally, since there are two top elements and they may appear in any order we must count the probability of Event B occurring twice.

\end{enumerate}


Overall we get:
\begin{equation}
    \frac{t(t - 1)}{j (j - 1)} + 2 \sum_{p = t + 1}^{j}\frac{1}{j}\frac{t}{j - 1}\frac{t-1}{p-2}
\end{equation}
Total probability:
\begin{align*}
    \frac{1}{n} \sum_{j = t}^{n-1}\left(\frac{t(t - 1)}{j (j - 1)} + 2 \sum_{p = t + 1}^{j}\frac{1}{j}\frac{t}{j - 1}\frac{t-1}{p-2}\right) 
    &= 
   \frac{1}{n}\sum_{j = t}^{n-1}\left(1 + 2\sum_{p=t}^{j - 1} \frac{1}{p - 1} \right) \\
   &= 
   \frac{t(t - 1)}{n}\sum_{j=t}^{n-1}\left(\frac{1}{j(j-1)} + \frac{2}{j(j-1)}\sum_{p = t}^{j-1}\frac{1}{p-1}\right)\\
   & >
   \frac{t(t-1)}{n}\sum_{j = t}^{n - 1}\left(\frac{1}{j^2} + 2\frac{1}{j^2}\sum_{p = t + 1}^{j}\frac{1}{p-1}\right) \\
   & >
   \frac{t(t-1)}{n}\sum_{j = t}^{n - 1}\left(\frac{1}{j^2} + \frac{2}{j^2}\int_{p=t+1}^{j+1} \frac{1}{p-1}\, dp \right)\\
   & >
   \frac{t(t-1)}{n}\sum_{j = t}^{n-1}\left(\frac{1}{j^2} + \frac{2}{j^2}\ln\left(\frac{j}{t}\right)\right)
%   =\frac{t(t-1)}{n}(\sum_{j = t}^{a}(\frac{1}{j^2} + \frac{2}{j^2}\ln(\frac{j}{t})) + \sum_{j = a + 1}^{n - 1}(\frac{1}{j^2} + \frac{2}{j^2}\ln(\frac{j}{t})))\\
%   &\textgreater
%   \frac{t(t-1)}{n}(\int)
%   | \int_n^{n+1} f(t) dt - f(n)| \leq \int |f(t) - f(n)| dt \leq \int sup_{[n,t]}|f'(x)||t-n| dt
\end{align*}
Now we will use the following lemma
\begin{lemma}
For any differentiable function $f$ and any $a<b$, we have,
\begin{equation}
    \sum_{j=a}^{b} f(j) \geq \int_a^{b+1} f(t) dt - |b+1-a|\sup_{t \in [a,b+1]} |f'(t)|
\end{equation}
\end{lemma}
\begin{proof}
\begin{equation}
     | f(n) - \int_n^{n+1} f(t) dt | \leq \int_n^{n+1} |f(n)-f(t)| dt \leq \sup_{t \in [n,n+1]} |f'(t)| 
\end{equation}
Thus, 
\begin{equation}
    f(n) \geq  \int_n^{n+1} f(t) dt - \sup_{t \in [n,n+1]} |f'(t)| 
\end{equation}
and by summing for $n = a \ldots b$ we get the desired lemma. 
\end{proof}
Applying this lemma to $f(x) = \frac{1+2\ln(x/t)}{x^2}\,, a = t$ and $b=n-1$, we get 
\begin{align}
    \frac{1}{n} \sum_{j = t}^{n-1}\frac{t(t - 1)}{j (j - 1)} + 2 \sum_{p = t + 1}^{j}\frac{1}{j}\frac{t}{j - 1}\frac{t-1}{p-2} 
    & > \frac{t(t-1)}{n}\sum_{j = t}^{n-1}\left(\frac{1}{j^2} + \frac{2}{j^2}\ln\left(\frac{j}{t}\right)\right) \\
    & \geq \frac{t(t-1)}{n}\left(\int_{t}^{n}  \frac{1+2\ln(x/t)}{x^2} dx - 2(n-t) \sup_{x \in [t,n]} \left| 
    \frac{4 \ln(x/t)}{x ^3}\right| \right) \notag \\
    & \geq  \frac{t(t-1)}{n}\left(\int_{t}^{n}  \frac{1+2\ln(x/t)}{x^2} dx - 2(n-t) \left| \frac{16}{3 t^3 e^4} \right| \right) \notag \\
    & = 
    \frac{t(t -1)}{n} \left(\frac{3}{t}-\frac{2\ln(n/t) + 3}{n} - 2(n-t) \left| \frac{16}{3 t^3 e^4} \right|\right) 
\end{align}
Now for $t = \alpha n$ where $\alpha \in (0, 1)$ and as $n \xrightarrow[]{} \infty$, that lower-bound becomes 
\begin{equation}
C \geq \alpha (3 - \alpha(3 - 2\ln (\alpha))) + \mathcal{O}(1/n) \,,\quad \forall \alpha \in (0,1)
\end{equation}
The constant term of the RHS is a concave function of $\alpha$ that is maximized for $\alpha^* \approx 0.38240$. Thus, our algorithms achieves competitive ratio larger than $0.42737$.
\end{proof}


\clearpage

\section{Competitive Ratio General $k$}
\label{app:general_k_proof}
We now prove our main result for the competitive ratio of \algoname \ for $k \geq 2$. The theorem statement is reproduced here for convenience.  
\begin{reptheorem}{thm:general_k_theorem1}
The competitive ratio of Virtual Plus for the case $k \geq 2$ where $t = \alpha n$ can asymptotically be lower bounded by 
\begin{equation}
    C_k >  \max_{\alpha \in [0,1]}  {\alpha}^k \left(\sum_{m = 0}^{k - 1} a_m \ln^m (\alpha)\right) - \alpha a_0
    \hspace{0.15cm }where  \hspace{0.15cm }
    a_m = \left(\frac{\frac{k^k}{(k-1)^{k-m}} - k^m}{m!}\right)(-1)^{m+1}
    \, .
\end{equation}

\end{reptheorem}
\begin{figure}[ht]
    \centering
    %\includegraphics[width=0.50\linewidth]{Figures/virtual_plus_general_k.pdf}
    \includegraphics[width=1.0\linewidth]{Figures/general_k.pdf}
    %\includegraphics[width=\linewidth]{Figures/Competitive_RatioVar5.png}
    \caption{Virtual+ $k \geq 2$ proof.}
    \label{fig:general_k}
    \vspace{-15pt}
\end{figure}
\begin{proof}
First note that we can show that the competitive ratio for the $k$-secretary problem is equal to 
\begin{equation}
    C = \frac{1}{k}\sum_{a=1}^k \mathbb{P}(i_a \in S_\mathcal{A}), \label{eq:C_as_sum_prob1}
\end{equation}
where $i_a$ is the index of the $a^{th}$ secretary picked by the offline solution ---i.e. $i_a$ is a top-$k$ secretary of $\mathcal{D}$. 

\begin{align}
    \mathbb{P}(i_a \in S_\mathcal{A}) 
    &= \sum_{j=t}^{n-1} \mathbb{P}(i_a \in S_\mathcal{A} \text{ at time-step }j+1) \label{p_picked_equal_not_filled} \\
    &= \frac{1}{n}\sum_{j=t}^{n-1} \mathbb{P}(|S_{\mathcal{A}}| < k \text{ at time-step } j + 1) \notag
\end{align}

Now, we compute $\mathbb{P}(|S_{\mathcal{A}}| < k \text{ at time-step } j+1)$ by decomposing this probability into the smaller events aka $\mathbb{P}(|S_{\mathcal{A}}| = 1 \text{ at time-step } j+1)$, $\mathbb{P}(|S_{\mathcal{A}}| = 2 \text{ at time-step } j+1)\dots \mathbb{P}(|S_{\mathcal{A}}| = k - 1 \text{ at time-step } j+1)$

We compute the probability of $\mathbb{P}(|S_{\mathcal{A}}| = \nu \text{ at time-step } j+1)$ in the following manner. First of let's say that $\nu$ elements are being selected by the algorithm at the time steps $p_1$, $p_2, \dots p_{\nu}$. In order for element to be selected at position $p_\nu$ that element has to be one of the top $k$ elements up to $j+1$. Therefore we have factor $\frac{k}{j}$ in our equation. Now, in order to make sure that none of the elements are picked after the position $p_\nu$ we need to make sure that the rest of top - $k$ up to $j+1$ appeared before $p_\nu$ therefore we have the factor of $\frac{{p_{\nu} - 1 \choose k - 1}}{{j-1 \choose k - 1}}$. Similarly we calculate recursively for each position $p_{\nu-1} \dots p_1$. However, we also need to make sure that none of the elements are picked in the time interval $[t+1 \dots p_1-1]$
aka before $p_1$ and probability for that is  $\frac{{t \choose k}}{{p_1-1 \choose k}}$ - when the top $k$ elements up to $p_1 - 1$ appear in the sampling phase.
Probability of $\mathbb{P}(|S_{\mathcal{A}}| = \nu \text{ at time-step } j+1)$ is :
\begin{align}
    &= \sum_{t+1 \leq p_1 < p_2 < \dots <p_{k-1} \leq j}\frac{k}{j} \frac{{p_{\nu} - 1 \choose k-1}}{{j-1 \choose k-1}}\frac{k}{p_{\nu} - 1}
    \frac{{p_{{\nu}-1} - 1 \choose k-1}}{{p_{\nu} - 2 \choose k-1}}\frac{k}{p_{\nu-1} - 1}\dots \frac{k}{p_2 - 1}\frac{{p_1 - 1 \choose k - 1}}{{p_2 - 2 \choose k-1}} \frac{{t \choose k}}{{p_1-1 \choose k}}\\
    & = \frac{t(t-1)\dots(t-k+1)}{j(j-1)\dots (j - k + 1)}\sum_{t+1 \leq p_1 < p_2 < \dots <p_{k-1} \leq j}
    \frac{k^{\nu}}{(p_\nu-k)(p_{\nu-1} - k)\dots(p_1 - k)}
\end{align}

Therefore, probability of not being overfilled before time step j + 1:
\begin{align}
    & \frac{t(t - 1)...(t - k + 1)}{j (j - 1) \dots (j - k + 1)} \Big(1 + k\hspace{-1em}\sum_{p_1 = t + 1\dots j}^{}\frac{1}{p_1 - k} + \dots
    + {k^{k - 1}}\hspace{-2em}\sum_{\substack{p_1 = t + 1 \dots j \\\vdots\\ p_{k-1} = t + 1 \dots p_{k-2}}}\frac{1}{(p_1 - k)\dots (p_{k-1} - k)} \Big)\\
    & = 
    \frac{t(t - 1)...(t - k + 1)}{j (j - 1) \dots (j - k + 1)} \Big(1 + k\hspace{-0.5 em}\sum_{p_1 = t + 1}^{j}\frac{1}{p_1 - k} + 
     \dots  +
     {k^{k - 1}}\sum_{p_1 = t + 1}^{j}\frac{1}{ (p_{1} - k)} \dots \sum_{p_{k-1} = t + 1}^{p_{k-1}}\frac{1}{ (p_{k-1} - k)} \Big) 
\end{align}

Now we make use of the  
Lemma~\ref{lemma_two}:
\begin{align}
 = 
     \frac{t(t - 1)\dots (t - k + 1)}{j(j - 1)\dots(j - k + 1)}(1 + {k}\frac{1}{1!}\ln (\frac{j - k}{t}) + {k^2}\frac{1}{2!}\ln^2(\frac{j - k}{t}) + \dots + {k^{k - 1}}\frac{1}{(k-1)!}\ln^{k - 1}(\frac{j - k}{t})\Big)
\end{align}

Then we get that the total competitive ratio: 
\begin{align}
     & \frac{1}{n}\sum_{j = t}^{n - 1}
     \frac{t(t - 1)\dots (t - k + 1)}{j(j - 1)\dots(j - k + 1)}(1 + {k}\frac{1}{1!}\ln (\frac{j - k}{t}) + {k ^ 2}\frac{1}{2!}\ln^2(\frac{j - k}{t}) + \dots + {k^{k - 1}}\frac{1}{(k-1)!}\ln^{k - 1}(\frac{j - k}{t})\Big)\\
     & \geq
     \frac{1}{n}\int_{j = t}^{n}
     \frac{t(t - 1)\dots (t - k + 1)}{j(j - 1)\dots(j - k + 1)}\Big(1 + {k}\frac{1}{1!}\ln (\frac{j - k}{t}) + {k^2}\frac{1}{2!}\ln^2(\frac{j - k}{t}) + \dots + {k^{k - 1}}\frac{1}{(k-1)!}\ln^{k - 1}(\frac{j - k}{t})\Big)\\
     & \geq
     \frac{1}{n}\int_{j = t}^{n}
     \frac{t(t - 1)\dots (t - k + 1)}{j^k}\Big(1 + {k}\frac{1}{1!}\ln (\frac{j - k}{t}) + {k^2}\frac{1}{2!}\ln^2(\frac{j - k}{t}) + \dots + {k^{k - 1}}\frac{1}{(k-1)!}\ln^{k - 1}(\frac{j - k}{t})\Big)\\
\end{align}
Now notice that:
\begin{equation}
    \label{identity_two}
    \int\frac{1}{a!} \frac{\ln^a(x)}{x^k} dx = - \frac{1}{x^{k - 1}}\sum_{m = 0}^{a}\frac{1}{m!}(k -1)^{m - 1 -a }\ln^m(x)
\end{equation}
Then we make use of this identity and get that competitive ratio is:
\begin{align}
     & \geq\frac{t(t - 1)\dots (t - k + 1)}{n}
     \Big(\sum_{a = 0}^{k - 1}- \frac{1}{j^{k - 1}}{ k ^ a }\sum_{m = 0}^{a}\frac{1}{m!}(k -1)^{m - 1 -a }\ln^m(\frac{j - k}{t}) \Big) \Big|_{j = t}^{n}\\
     & =
     \frac{t(t - 1)\dots (t - k + 1)}{n}\Big(- \frac{1}{j^{k - 1}}\sum_{m = 0}^{k - 1}\frac{1}{m!}\Big(\sum_{a = m}^{k - 1}{k ^ a}(k - 1)^{m - a - 1}\Big)\ln^m{(\frac{j - k}{t})}\Big)\Big|_{j = t}^{n}
\end{align}
Now as $t = \alpha n $ where $\alpha \in (0,1)$ and as $n \xrightarrow{}\infty$ our competitive rate becomes:
\begin{align}
    & \alpha \Big(\sum_{a = 0}^{k - 1}{ k ^ a }{(k - 1)}^{-1 - a}\Big) - \alpha^k\Big(\sum_{m=0}^{k - 1}\frac{1}{m!}\Big(\sum_{a = m}^{k - 1}{ k ^ a }(k - 1)^{m - a - 1}\Big)\ln^m(\frac{1}{\alpha})\Big)\\
    & = \alpha\left({\left(\frac{k}{k-1}\right)}^{k} - 1\right) - \alpha^k\left(\sum_{m-0}^{k-1}\left(\frac{\frac{k^k}{(k-1)^{k-m}} - k^m}{m!}\right)(-1)^{m+1}\ln^m(\alpha)\right)
\end{align}

\begin{lemma} 
\label{lemma_two}
Let $f_i \,, \,i = 1 \ldots k$ be decreasing positive functions then we have 
\begin{equation}
    \sum_{p_1=a_1}^{b_1} \ldots \sum_{p_k=a_k}^{p_{k-1}} f_1(p_1) \ldots f_k(p_k)
    \geq \int_{x_1=a_1}^{b_1+1} \ldots \int_{x_k = a_k}^{x_{k-1}+1}  f_1(x_1) \ldots f_k(x_k) dx_1\dots dx_k
\end{equation}
\end{lemma}
\begin{proof}
    The proof consists in noticing that since the functions $f_i \,, \,i = 1 \ldots k$ are decreasing and positive we have, 
    \begin{equation}
        f_1(p_1) \ldots f_k(p_k) \geq  f_1(p_1) \ldots  f_{k-1}(p_{k-1})\int_{x_k = p_k}^{p_k+1} f_k(x_k) dx_k
    \end{equation}
    Thus, by summing this inequality for $p_k= a_k \ldots p_{k-1}$, we get
    \begin{equation}
    \sum_{p_k=a_k}^{p_{k-1}} f_1(p_1) \ldots f_k(p_k)
    \geq  f_1(p_1) \ldots  f_{k-1}(p_{k-1})  \int_{x_k = a_k}^{p_{k-1}+1} f_k(x_k) dx_k
    \end{equation}
    Now, because the functions $f_i \,, \,i = 1 \ldots k$ are decreasing and positive we have, 
     \begin{align}
    \sum_{p_k=a_k}^{p_{k-1}} f_1(p_1) \ldots f_k(p_k)
    \geq  f_1(p_1) \ldots  f_{k-2}(p_{k-2})  \int_{x_{k-1} = p_{k-1}}^{p_{k-1}+1} f_{k-1}(x_{k-1})  \int_{x_k = a_k}^{p_{k-1}+1} f_k(x_k) dx_{k-1}dx_k \\
    \geq f_1(p_1) \ldots  f_{k-2}(p_{k-2})  \int_{x_{k-1} = p_{k-1}}^{p_{k-1}+1} f_{k-1}(x_{k-1})  \int_{x_k = a_k}^{x_{k-1}} f_k(x_k) dx_{k-1}dx_k 
    \end{align}
    where for the last inequality we used the fact that $x_{k-1} \in [p_{k-1}, p_{k-1}+1]$.
    Finally, by summing for $p_{k-1} = a_{k-1} \ldots p_{k-2}$, we get,
    \begin{equation}
    \sum_{p_{k-1}=a_{k-1}}^{p_{k-2}}  \sum_{p_k=a_k}^{p_{k-1}} f_1(p_1) \ldots f_k(p_k)
    \geq f_1(p_1) \ldots  f_{k-2}(p_{k-2})  \int_{x_{k-1} = p_{k-1}}^{p_{k-1}+1} f_{k-1}(x_{k-1})  \int_{x_k = a_k}^{x_{k-1}} f_k(x_k) dx_{k-1}dx_k
    \end{equation}
    With a recursive argument, we finally get,
    \begin{equation}
    \sum_{p_1=a_1}^{b_1} \ldots \sum_{p_k=a_k}^{p_{k-1}} f_1(p_1) \ldots f_k(p_k)
    \geq \int_{x_1=a_1}^{b_1+1} \ldots \int_{x_k = a_k}^{x_{k-1}+1}  f_1(x_1) \ldots f_k(x_k) dx_1\dots dx_k
\end{equation}
\end{proof}

\end{proof}

\subsection{Table of $C_k$}

\begin{table}[]
    \centering
    \caption{Values of the Competitive ratio $C_k$ and the associated optimal threshold $\alpha_k$ for \algoname. Note that for $5\leq k\leq 100$ the competitive ratio of \textsc{Single-Ref} provided by~\citet{albers2020new} outperforms \algoname's competitive ratio. However, our analysis provides a tractable way for the competitive ratio that scales with $k$ since the computation of the function to optimize (and its gradients) in Theorem~\ref{thm:K_2_theorem1} is a $\mathcal{O}(k)$.}
    \vspace{2pt}
    \begin{tabular}{ccccccccccc}
     \toprule
          $k$ & 2 &3 & 4 & 5 & 100 & 200 &300 &400 & 500 & 600\\
         \midrule
        $C_k$ & .4273&.4575&.4769&.4906&.5959&.6062&.6108&.6136&.6156&.6170 \\
        $t_k$ & .3824 &.3867&.3884&.3890&.3781 &.3755 &.3743&.3735&.3729&.3726 \\
        \bottomrule
    \end{tabular}
    \label{tab:C_k}
\end{table}

\section{Proof of Theorem~\ref{thm:stochastic_secretary}}
\label{appendix:proof_thm2}
We now prove Theorem \ref{thm:stochastic_secretary} in detail, reproduced here for convenience. 

\begin{reptheorem}{thm:stochastic_secretary}
Let us consider a secretary algorithm, $\mathcal{A}$, that is $C_n$-competitive when given access to the true values $v_i, \,i \in [n]$. When having access to independent random variables $\mathcal{V}_{i},\, i\in [n]$ such that Eq.~\ref{eq:concentration} holds, $\mathcal{A}$ has a stochastic competitive ratio of at least:  
\begin{equation}
    C_n \left( 1-e^{- \frac{\Delta}{2 \sigma^2}} \right)^{\frac{-2 \exp(- \frac{\Delta}{2 \sigma^2})}{1 -  \exp(\frac{-\Delta}{\sigma^2})}}
\end{equation}
\end{reptheorem}

\begin{proof}
In this analysis, we will refer to a deterministic online algorithm as $\mathcal{A}$. Please note, the following statement of results holds for \textit{any} online algorithm, therefore when introducing certain lemmas we will use algorithm $\mathcal{A}$ for the name. Furthermore, we will denote the stochastic versions of $\mathcal{A}$ as $\mathcal{A}^{s}$.
Let $v_1^*, v_2^*, \dots v_k^*$ be the values of top $k$ elements and let $i_1^*, i_2^* \dots i_k^*$ be their appropriate indices. Therefore, the optimal offline solutions selects set 
$S^* = \{i_1^*, i_2^*, \dots i_k^*\}$, while a deterministic online algorithm $\mathcal{A}$ chooses set $S$, and our online algorithm $\mathcal{A}^{s}$ chooses set $S^{(s)}$. Our goal is to provide a lower bound of the kind
\begin{equation}
     \frac{\mathbb E[|S^{(s)} \cap S^*|]}{k} \geq c\frac{\mathbb E[|S \cap S^*|]}{k}
\end{equation}
where $c>0$ is a constant that depends on the randomness of the estimates $\mathcal{V}_i\,,i=1,\ldots,n.$ 
Assuming that $\mathcal{A}$ is $C_n$-optimal, this lower bound will provide us a competitive ratio for the stochastic algorithm $\mathcal{A}^{s}$.

% 
% Note that for any two random variable $X_i$ and $X_j$ for which we know that $\omega_i > \omega_j$ holds:
% \begin{align*}
% \sP(X_i < X_j)  & = 
% \sP((X_i - \omega_i) - (X_j - \omega_j) < \omega_j - \omega_i)  \\ & =   \sP (\epsilon_i - \epsilon_j < -\Delta) \leq e^{\frac{-\Delta}{{2\sigma}^2}}
% \end{align*}
%  therefore
%  \begin{equation}\label{equ:eight}
%  \sP(X_i > X_j) \geq (1 -  e^{\frac{-\Delta}{{2\sigma}^2}})
% \end{equation}


Now, by definition, the algorithm $\mathcal{A}^{s}$, is the algorithm $\mathcal{A}$ but operating on the random variables $\mathcal{V}_1, \mathcal{V}_2, ...\mathcal{V}_n$ (instead of $v_1,\ldots, v_n$. Thus, is we call $i_a^{(s)}$ the index of the $a$ largest value among  $\mathcal{V}_1, \mathcal{V}_2, ...\mathcal{V}_n$ (note that $i_a^{(s)}$ is a random variable) we have that 
\begin{equation}
   \sP[i_a^* \in S] = \sP[i_a^{(s)} \in S^{(s)}|\mathcal{V}_1, \dots,\mathcal{V}_n]
\end{equation}
and thus, by summing over $a$ and taking expectation on all permutations and noting $\{i_1^{(s)},\ldots,i_k^{(s)}\} := S^*_s$
\begin{equation}
    \sE[|S^* \cap S|] = \sE[|S^*_s \cap S^{(s)}|  \; | \mathcal{V}_1, \ldots,\mathcal{V}_n]
\end{equation}
Now let us note that 
we have that the expected number indices in $S^*$ picked by $\mathcal{A}^{s}$ is larger than the expected number of indices in $S^*_s\cup S^*$, formally,
\begin{align}
    \sE[|S^* \cap S^{(s)}|]  \notag
    &\geq \sE[|S^* \cap S^{(s)} \cap S^*_s|] \notag \\
    &\geq \sE [ \mathbf{1}\{S^* = S^*_s\} \sE[|S^* \cap S^*_s|\,| \mathcal{V}_1, \ldots,\mathcal{V}_n]] \notag\\
    &= \sP(S^*=S^*_s) \sE[|S^* \cap S|]
\end{align}
% \begin{equation}
%      \sum_{a=1}^k \sP[i_a^* \in S^{(s)}]
%      \geq \sum
%      _{a=1}^k \sP[i_a^{(s)} \in S^{(s)} \text{ and } i_a^{(s)} \in S^*]
% \end{equation}
Now, notice that without loss of generality we can consider that $v_1 \geq \dots \geq v_n$. Let us call, $\bar v:= \frac{v_k + v_{k+1}}{2}$, if we have $\mathcal{V}_{i_a^*} \geq \bar v \geq \mathcal{V}_{i_b^*}$ for $1 \leq a \leq k < b \leq n$ then we will have $S^*=S^*_s$. Let us formally bound this probability,
\begin{align*}
    \sP(S^*=S^*_s) 
    &\geq \sP(\mathcal{V}_{i_1^*}, \ldots, \mathcal{V}_{i_k^*} \geq \ldots \geq \mathcal{V}_{i_{k+1}^*} , \ldots, \mathcal{V}_{i_n^*})  \\
     & \geq \sP(\mathcal{V}_{i_a^*} \geq \bar v \geq \mathcal{V}_{i_b^*}\,,\,1 \leq a \leq k < b \leq n) \\
    & \geq \sP(\mathcal{V}_{i_a^*} \geq v_a - (2(k-a)+1)\Delta\,,\,   v_b + (2(b-k)-1) \Delta \geq  \mathcal{V}_{i_b^*}\,,\,1 \leq a \leq k < b \leq n) \\
\end{align*}
where is the last inequality we used that because $\Delta = \tfrac{1}{2}\min_{1\leq i \neq j \leq  n } \{\mid v_i - v_j \mid \}
$ we have
\begin{equation}
    v_a - (2(k-a)+1)\Delta \geq \bar v \geq  v_b + (2(b-k)-1) \Delta \,, \quad \forall \, 1 \leq a \leq k < b \leq n \,.
\end{equation}
Finally, using that the random variable $\mathcal{V}_i$ are independents we have,
\begin{align*}
    \sP(S^*=S^*_s)      
    & \geq \sP(|\mathcal{V}_{i_a^*} - v_a| \leq  2|k+1/2-a|\Delta, \,a \in [n]) \\
    & = \prod_{a=1}^n \sP(|\mathcal{V}_{i_a^*} - v_a| \leq  2|k+1/2-a|\Delta) \\
    & \geq \prod_{i=1}^n  (1 - e^{\frac{-2|k+1/2-i|\Delta}{2\sigma^2}})
\end{align*}
Where for the last inequality we used the assumption that 
\begin{equation}
    \sP( |\mathcal{V}_i-v_i| \geq \epsilon) \leq e^{\frac{-\epsilon^2}{2\sigma^2}} \,,\,\forall i \in [n]\,.
\end{equation}

Finally, we will lower-bound $\sum_{i=1}^n\log (1 - e^{\frac{-2|k+1/2-i|\Delta}{2\sigma^2}})$ by using the fact that for any $x$ such that $0<x<a<1$, we have by concavity of the $\log$,
\begin{equation}
    \log(1-x) \geq \log(1-a) x \,.
\end{equation}
Thus,
\begin{align}
    \sum_{l=1}^n\log (1 - e^{\frac{-2|k+1/2-l|\Delta}{2\sigma^2}}) 
    & \geq \log(1-e^{- \frac{\Delta}{2 \sigma^2}}) \sum_{l=1}^n e^{\frac{-2|k+1/2-l|\Delta}{2\sigma^2}} \\
    & \geq -2\log(1-e^{- \frac{\Delta}{2 \sigma^2}}) e^{- \frac{\Delta}{2 \sigma^2}} \sum_{l=1}^{\infty} e^{\frac{-l\Delta}{\sigma^2}} \\
    & \geq -2\log(1-e^{- \frac{\Delta}{2 \sigma^2}}) \frac{e^{- \frac{\Delta}{2 \sigma^2}} }{1 -  e^{\frac{-\Delta}{\sigma^2}}}
\end{align}
Which finally gives us,
\begin{equation}
     \sP(S^*=S^*_s)    \geq \left( 1-e^{- \frac{\Delta}{2 \sigma^2}} \right)^{\frac{-2 \exp(- \frac{\Delta}{2 \sigma^2})}{1 -  \exp(\frac{-\Delta}{\sigma^2})}}
\end{equation}



Note that we could refine this bound by considering individual gaps $\Delta_i$ and constants $\sigma_i$ and consider $\frac{\Delta}{\sigma^2} = \min_{i \in [n]} \frac{\Delta_i}{\sigma_i}$.

\end{proof}


\clearpage


\section{Classical Online Algorithms for Secretary Problems}
\label{appendix:classical_online_algorithms}

All single threshold online algorithm described in this paper include: \textsc{Virtual}, \textsc{Optimistic} and \textsc{Single-Ref}. Each online algorithm consists of two phases ---\textbf{sampling phase} followed by \textbf{selection phase}--- and an optimal stopping point $t$ which is used by the algorithm to transition between the phases. We now briefly summarize these two phases for the aforementioned online algorithms. 

\xhdr{Sampling Phase - \textsc{Virtual}, \textsc{Optimistic} and \textsc{Single-Ref}}
In the sampling phase, the algorithms passively observe all data points up to a pre-specified time index $t$, but also maintains a sorted reference list $R$ consisting of the $k$ elements with the largest values $\mathcal{V}(i)$ seen. Thus the $R$ contains a list of elements sorted by decreasing value. That is $R[k]$ is the index of the $k$-th largest element in $R$ and $\mathcal{V}(R[k])$ is its corresponding value. The elements in $R$ are kept for comparison but are crucially \textit{not} selected in the sampling phase.

\subsection{\textsc{Virtual} Algorithm}

\xhdr{Selection Phase - \textsc{Virtual} algorithm}
Subsequently, in the selection phase, $i > t$, when an item with value $\mathcal{V}(i)$ is observed an irrevocable decision is made of whether the algorithm should select $i$ into $S$. To do so, the Virtual algorithm simply checks if the value of the $k$-th smallest element in $R$, $\mathcal{V}(R[k])$, is smaller than $\mathcal{V}(i)$ in addition to possibly updating the set $R$. The full Virtual algorithm is presented in Algorithm 1.

\begin{algorithm}[ht]
\textbf{Inputs:} $t\in[k\dots n-k]$, $R = \emptyset$, $S_{\mathcal{A}} = \emptyset$
\newline
\textbf{Sampling phase:} Observe the first $t$ data points and construct a list $R$ with the indices of the top $k$ data points seen.  $\texttt{sort}$ ensures: $ \mathcal{V}(R[1]) \geq \mathcal{V}(R[2]) \dots \geq \mathcal{V}(R[k]).$

\textbf{Selection phase (at time $i>t$):}

\begin{algorithmic}[1]
\IF {$ \mathcal{V}(i) \geq  \mathcal{V}(R[k])$
and $R[k] > t$} 
        \STATE $R$ = $\texttt{sort}\{R \cup \{i\} \setminus \{R[k]\}\}$ \hfill\COMMENT{// Update $R$ with element $i$ and also take out $R[k]$}
\ELSIF{ $ \mathcal{V}(i) \geq  \mathcal{V}(R[k])$
and $R[k] \leq t$}
        \STATE $R$ = $\texttt{sort}\{R \cup \{i\} \setminus \{R[k]\}\}$ \hfill\COMMENT{// Update $R$ with element $i$ and also take out $R[k]$}
        \STATE $S_\mathcal{A} = \{ S_\mathcal{A} \cup \{i \}\}$ \hfill\COMMENT{// Select element $i$}
\ENDIF 
\STATE
$i\gets i + 1$
\end{algorithmic}
 \caption{\textsc{Virtual Algorithm}}
\end{algorithm}


\subsection{\textsc{Optimistic} Algorithm}

\xhdr{Selection Phase - \textsc{Optimistic} algorithm}
In the optimistic algorithm, i is selected if and only if $\mathcal{V}(i) \geq \mathcal{V}(R[last])$. Whenever $i$ is selected, $R[last]$
is removed from the list
$R$, but no new elements are ever added to $R$. Thus, intuitively, elements are selected when they beat one of the remaining reference
points from $R$.
We call this algorithm “optimistic” because it removes the reference
point $R[last]$ even if $ \mathcal{V}(i)$ exceeds, say, $ \mathcal{V}(R[1])$. Thus, it implicitly assumes
that it will see additional very valuable elements in the future, which
will be added when their values exceed those of the remaining, more
valuable, $R[a]\,,\,a \in [k]$.


\begin{algorithm}[ht]
\textbf{Inputs:} $t\in[k\dots n-k]$, $R = \emptyset$, $S_{\mathcal{A}} = \emptyset$.

\textbf{Sampling phase (up to time $t$):} Observe the first $t$ data points and construct a list $R$ with the indices of the top $k$ data points seen. $\texttt{sort}$ ensures: $ \mathcal{V}(R[1]) \geq \mathcal{V}(R[2]) \dots \geq\mathcal{V}(R[k]).$
Set $last = k$, to be the index of the last element in $R$.

\textbf{Selection phase (at time $i>t$):}

\begin{algorithmic}[1]
\IF {$\mathcal{V}(i) \geq \mathcal{V}(R[last])$} 
        \STATE $R$ = $\{R \setminus \{R[last]\}\}$ \hfill\COMMENT{// Update $R$ by taking out $R[k]$}
        \STATE $S_\mathcal{A} = \{ S_\mathcal{A} \cup \{i \}\}$ \hfill\COMMENT{// Select element $i$}
        \STATE $last = last - 1$
\ENDIF 
$i\gets i + 1$
\end{algorithmic}
 \caption{\textsc{Optimistic Algorithm}}
\end{algorithm}

\subsection{\textsc{Single-Ref} Algorithm}
\xhdr{Selection Phase - \textsc{Single-Ref} algorithm}
In the \textsc{Single-Ref} algorithm, $i$ is selected if and only if $\mathcal{V}(i) \geq \mathcal{V}(R[r])$ and we haven't already selected $k$ elements. 
We call this algorithm single reference algorithm because we always compare incoming elements to one single reference element, that was determined in the sampling phase.

\begin{algorithm}[ht]
\textbf{Inputs:} $t\in[k\dots n-k]$, $R = \emptyset$, $S_{\mathcal{A}} = \emptyset$, $r \in [k]$ (reference rank)

 \textbf{Sampling phase (up to time $t$):} Observe the first $t$ data points and construct a list $R$ with the indices of the top $k$ data points seen. Let $s_r = R[r]$ be the $r$-th best item from the sampling phase.

\textbf{Selection phase (at time $i>t$):} 
\begin{algorithmic}[1]
\IF {$\mathcal{V}(i) \geq s_r$ and $ |S_{\mathcal{A}}| \leq  k$} 
        \STATE $S_\mathcal{A} = \{ S_\mathcal{A} \cup \{i \}\}$ \hfill\COMMENT{// Choose the first $k$ items better than $s_r$}

\ENDIF 

$i\gets i + 1$
\end{algorithmic}
 \caption{\textsc{Single-Ref Algorithm}}
\end{algorithm}

\clearpage
\section{Additional Experimental Results}
\label{appendix:additional_results}

\subsection{Additional Results on Synthetic Data}
\label{appendix:synthetic_additional_results}
We now provide additional results on Synthetic Data with varying levels of noise added to each item in $\mathcal{D}$. In particular, we investigate in figure \ref{fig:additional_results_synthetic_data} online algorithms in the face of no noise ---i.e. $\sigma^2 = 0$, $\sigma^2 = 1$, and $\sigma^2 = 5$ in addition to $\sigma^2 = 10$ reported in figure \ref{fig:synthetic_data}. The deterministic setting corresponds to $\sigma^2 = 0$ while $\sigma^2=1$ and $\sigma^2=1$ correspond to the stochastic setting as introduced in section \ref{stochastic_k_secretary}.

\begin{figure}[ht]
    \centering
    \includegraphics[width=0.32\linewidth]{Figures/Competitive_RatioBar8-Deterministic.png}
    \includegraphics[width=0.32\linewidth]{Figures/Competitive_RatioBar8-Var-1.png}
    \includegraphics[width=0.32\linewidth]{Figures/Competitive_RatioBar-Var-5.png}
    \caption{Estimation of the competitive ratio of online algorithms under various noise levels. \textbf{Left:} Deterministic setting with $\sigma^2=0$. \textbf{Middle:} Stochastic setting with $\sigma^2 = 1$. \textbf{Right:} Stochastic setting with $\sigma^2 = 5$.}
    \label{fig:additional_results_synthetic_data}
\end{figure}




\subsection{Experimental Details}
\label{appendix:additional_results_larger_datasets}
We provide more details about the experiments presented in section~\ref{section:experiments}. For further details we also invite the reader to look at the code provided with the supplementary materials.

\paragraph{Attack strategies} We use two different attack strategies the Fast Gradient Sign Method (FGSM) \citep{goodfellow2014explaining} and 40 iterations of the PGD attack \citep{madry2017towards} with $l_\infty$.

\paragraph{Hyper-parameters of online algorithms} All the online algorithms except \textsc{Single-Ref} have a single hyper-parameters to choose which is the length of the sampling phase $t$. For \textsc{Virtual} and \textsc{Optimistic} we use $t=\floor{ \frac{t}{e}}$ which is the value suggested by theory in \citet{babaioff2007knapsack}. For \textsc{\algoname} we use $t=\alpha n$ as in Eq.~\ref{eq:alpha_eqn} which is the value suggested by Theorem~\ref{thm:K_2_theorem1}. \textsc{Single-Ref} has two hyper-parameters to choose the threshold $t$ ($c$ in the original paper) and reference rank $r$. For $k=1...100$ the values are given in \citet{albers2020new} and are numerical solutions to combinatorial optimization problems. However, for $k = 1000$ no values are specified and we choose $c=0.13$ and $r=40$ through grid search. Indeed, these values may not be optimal ones but we leave the choice of better values as future work.

\paragraph{MNIST model architectures} For table~\ref{table:non_robust_table1}, $f_s$ and $f_t$ are chosen randomly from an ensemble of trained classifiers. The ensemble is composed of five different architectures described in table~\ref{appendix:mnist_ens_adv_training_table}, with 5 trained models per architecture.

\begin{table}[h]
    \centering
          
            \begin{tabular}{ccccc}
            \toprule
                   &  A  & B  & C & D \\
                    \midrule
                   & Conv(64, 5, 5) + Relu & Dropout(0.2) & Conv(128, 3, 3) + Tanh & \multirow{2}{*}{\shortstack{FC(300) + Relu \\ Dropout(0.5)}}\\
                   & Conv(64, 5, 5) + Relu &  Conv(64, 8, 8) + Relu &  MaxPool(2,2) & \\
                   & Dropout(0.25) & Conv(128, 6, 6) + Relu  & Conv(64, 3, 3) + Tanh & \multirow{2}{*}{\shortstack{FC(300) + Relu \\ Dropout(0.5)}} \\
                   & FC(128) + Relu & Conv(128, 6, 6) + Relu & MaxPool(2,2) & \\
                   & Dropout(0.5) &  Dropout(0.5)  &  FC(128) + Relu &  \multirow{2}{*}{\shortstack{FC(300) + Relu \\ Dropout(0.5)}}\\
                   & FC + Softmax &  FC + Softmax &  FC + Softmax & \\
                   & & & & \multirow{2}{*}{\shortstack{FC(300) + Relu \\ Dropout(0.5)}} \\
                   & & & & \\
                   & & & & FC + Softmax \\
            \bottomrule
            \end{tabular}
            \caption{The different MNIST Architectures used for $f_s$ and $f_t$}
            \label{appendix:mnist_ens_adv_training_table}
    \end{table}

\paragraph{CIFAR model architectures} For table~\ref{table:non_robust_table1}, $f_s$ and $f_t$ are chosen randomly from an ensemble of trained classifiers. The ensemble is composed of five different architectures: VGG-16 \citep{simonyan2014very}, ResNet-18 (RN-18) \citep{he2016deep}, Wide ResNet (WR) \citep{zagoruyko2016wide}, DenseNet-121 (DN-121) \citep{huang2017densely} and Inception-V3 architectures (Inc-V3) \citep{szegedy2016rethinking}, with 5 trained models per architecture.

\subsection{Additional metrics}
In addition to the results provided in table~\ref{table:non_robust_table1}, we also provide two other metrics here: the stochastic competitive ratio in table~\ref{appendix:comp_ratio_non_robust} and the knapsack ratio table~\ref{appendix:knap_ratio_non_robust}. Where the knapsack ratio is defined as the sum value of $S_{\mathcal{A}}$ ---i.e. the sum of total loss, as selected by the online algorithm divided by the value of $S^*$ selected by the optimal offline algorithm.
We observe that the competitive ratio is not always a good metric to compare the actual performance of the different algorithms, since sometimes the online algorithm with the best competitive ratio is not the algorithm with the best fool rate. The knapsack ratio on the other hand seems to be a much better proxy for the actual performance of the algorithms, this is due to the fact that we're interested in picking elements that have have a good chance to fool the target classifier but are not necessarily the best possible attack.

\begin{table*}[ht]
\begin{small}
\caption{Competitive ratio on non-robust models using FGSM and PGD attacker and various online algorithms.}
\label{appendix:comp_ratio_non_robust}
 \begin{center}\begin{tabular}{ c c c c c c c c }
 \toprule
 & & \multicolumn{3}{c}{MNIST (competitive ratio)} & \multicolumn{3}{c}{CIFAR-10 (competitive ratio)}\\
 & Algorithm & $k=10$ & $k=100$ & $k=1000$ & $k=10$ & $k=100$ & $k=1000$ \\
 \midrule
 \multirow{6}{*}{FGSM}
 & \textsc{Naive} (Lower-bound) & .006 $\pm$ .001 & .010 $\pm$ .000 & .098 $\pm$ .000 & .002 $\pm$ .000 & .010 $\pm$ .000 & .100 $\pm$ .000\\
 \cmidrule{2-8}
 & \textsc{Optimistic} & .063 $\pm$ .004 & .083 $\pm$ .003 & .197 $\pm$ .003 & .035 $\pm$ .002 & .064 $\pm$ .001 & .203 $\pm$ .001\\
 & \textsc{Virtual} & .048 $\pm$ .003 & .079 $\pm$ .003 & .201 $\pm$ .003 & .030 $\pm$ .002 & .073 $\pm$ .001 & .212 $\pm$ .001\\
 & \textsc{Single-Ref} & .070 $\pm$ .004 & .135 $\pm$ .006 & .181 $\pm$ .003 & .045 $\pm$ .002 & .109 $\pm$ .002 & .174 $\pm$ .001\\
 & \algoname & .072 $\pm$ .004 & .124 $\pm$ .005 & .270 $\pm$ .005 & .043 $\pm$ .002 & .107 $\pm$ .002 & .287 $\pm$ .002\\
 \midrule
 \multirow{6}{*}{PGD}
 & \textsc{Naive} (lower bound) & .005 $\pm$ .001 & .010 $\pm$ .000 & .098 $\pm$ .000 & .001 $\pm$ .000 & .010 $\pm$ .000 & .100 $\pm$ .000\\
 \cmidrule{2-8}
 & \textsc{Optimistic} & .023 $\pm$ .002 & .036 $\pm$ .001 & .156 $\pm$ .001 & .033 $\pm$ .002 & .052 $\pm$ .002 & .157 $\pm$ .002\\
 & \textsc{Virtual} & .011 $\pm$ .001 & .049 $\pm$ .001 & .173 $\pm$ .001 & .028 $\pm$ .002 & .056 $\pm$ .002 & .160 $\pm$ .002\\
 &\textsc{Single-Ref} & .032 $\pm$ .002 & .067 $\pm$ .002 & .135 $\pm$ .001 & .042 $\pm$ .003 & .087 $\pm$ .003 & .145 $\pm$ .001\\
 & \algoname & .023 $\pm$ .002 & .059 $\pm$ .002 & .215 $\pm$ .002 & .040 $\pm$ .002 & .081 $\pm$ .003 & .200 $\pm$ .003\\
 \bottomrule
\end{tabular}\end{center} 
\end{small}
\end{table*}

\begin{table*}[ht]
\begin{small}
\caption{Knapsack ratio on non-robust models using FGSM and PGD attacker and various online algorithms.}
\label{appendix:knap_ratio_non_robust}
 \begin{center}\begin{tabular}{ c c c c c c c c }
 \toprule
 & & \multicolumn{3}{c}{MNIST (knapscak ratio in \%)} & \multicolumn{3}{c}{CIFAR-10 (knapsack ratio in \%)}\\
 & Algorithm & $k=10$ & $k=100$ & $k=1000$ & $k=10$ & $k=100$ & $k=1000$ \\
 \midrule
 \multirow{6}{*}{FGSM}
 & \textsc{Naive} (Lower-bound) & 19.0 $\pm$ 0.3 & 19.5 $\pm$ 0.2 & 29.9 $\pm$ 0.2 & 16.8 $\pm$ 0.2 & 20.3 $\pm$ 0.1 & 28.7 $\pm$ 0.1\\
 \cmidrule{2-8}
 & \textsc{Optimistic} & 33.0 $\pm$ 0.6 & 33.1 $\pm$ 0.3 & 42.1 $\pm$ 0.3 & 32.7 $\pm$ 0.4 & 34.1 $\pm$ 0.2 & 42.8 $\pm$ 0.2\\
 & \textsc{Virtual} & 30.8 $\pm$ 0.5 & 34.2 $\pm$ 0.3 & 42.9 $\pm$ 0.3 & 32.9 $\pm$ 0.4 & 37.8 $\pm$ 0.2 & 45.0 $\pm$ 0.2\\
 & \textsc{Single-Ref} & 39.7 $\pm$ 0.6 & 41.5 $\pm$ 0.6 & 40.2 $\pm$ 0.3 & 37.5 $\pm$ 0.5 & 45.7 $\pm$ 0.4 & 37.9 $\pm$ 0.1\\
 & \algoname & 36.2 $\pm$ 0.6 & 41.1 $\pm$ 0.6 & 51.4 $\pm$ 0.5 & 39.4 $\pm$ 0.5 & 47.1 $\pm$ 0.4 & 55.5 $\pm$ 0.3\\
 \midrule
 \multirow{6}{*}{PGD}
 & \textsc{Naive} (lower bound) & 27.2 $\pm$ 0.6 & 15.5 $\pm$ 0.3 & 25.9 $\pm$ 0.3 & 22.5 $\pm$ 0.3 & 26.8 $\pm$ 0.2 & 36.3 $\pm$ 0.2\\
 \cmidrule{2-8}
 & \textsc{Optimistic} & 37.2 $\pm$ 0.9 & 24.2 $\pm$ 0.5 & 35.8 $\pm$ 0.4 & 35.3 $\pm$ 0.6 & 35.6 $\pm$ 0.4 & 43.1 $\pm$ 0.4\\
 & \textsc{Virtual} & 35.6 $\pm$ 0.9 & 27.5 $\pm$ 0.6 & 38.8 $\pm$ 0.4 & 35.5 $\pm$ 0.6 & 37.2 $\pm$ 0.5 & 43.9 $\pm$ 0.4\\
 &\textsc{Single-Ref} & 46.9 $\pm$ 1.1 & 34.3 $\pm$ 0.8 & 32.3 $\pm$ 0.5 & 39.0 $\pm$ 0.7 & 42.6 $\pm$ 0.6 & 41.2 $\pm$ 0.3\\
 & \algoname & 41.5 $\pm$ 1.1 & 32.3 $\pm$ 0.8 & 46.5 $\pm$ 0.6 & 40.9 $\pm$ 0.7 & 42.9 $\pm$ 0.6 & 48.8 $\pm$ 0.5\\
 \bottomrule
\end{tabular}\end{center} 
\end{small}
\end{table*}

\begin{table*}[ht]
\begin{small}
\caption{Competitive ratio on robust models using FGSM and PGD attacker and various online algorithms.}
\label{appendix:comp_ratio_robust}
 \begin{center}\begin{tabular}{ c c c c c c c c }
 \toprule
 & & \multicolumn{3}{c}{MNIST (competitive ratio)} & \multicolumn{3}{c}{CIFAR-10 (comptetitive ratio}\\
 & Algorithm & $k=10$ & $k=100$ & $k=1000$ & $k=10$ & $k=100$ & $k=1000$ \\
 \midrule
 \multirow{6}{*}{FGSM}
 & \textsc{Naive} (Lower-bound) & 0.00 $\pm$ 0.00 & 0.01 $\pm$ 0.00 & 0.10 $\pm$ 0.00 & 0.00 $\pm$ 0.00 & 0.01 $\pm$ 0.00 & 0.10 $\pm$ 0.00\\
 \cmidrule{2-8}
 & \textsc{Optimistic} & 0.24 $\pm$ 0.00 & 0.17 $\pm$ 0.00 & 0.33 $\pm$ 0.00 & 0.05 $\pm$ 0.00 & 0.21 $\pm$ 0.00 & 0.33 $\pm$ 0.00\\
 & \textsc{Virtual} & 0.18 $\pm$ 0.00 & 0.17 $\pm$ 0.00 & 0.33 $\pm$ 0.00 & 0.09 $\pm$ 0.00 & 0.22 $\pm$ 0.00 & 0.33 $\pm$ 0.00\\
 & \textsc{Single-Ref} & 0.27 $\pm$ 0.00 & 0.31 $\pm$ 0.00 & 0.28 $\pm$ 0.00 & 0.07 $\pm$ 0.00 & 0.39 $\pm$ 0.00 & 0.28 $\pm$ 0.00\\
 & \algoname & 0.25 $\pm$ 0.00 & 0.27 $\pm$ 0.00 & 0.49 $\pm$ 0.00 & 0.11 $\pm$ 0.00 & 0.35 $\pm$ 0.00 & 0.49 $\pm$ 0.00\\
 \midrule
 \multirow{6}{*}{PGD}
& \textsc{Naive} (Lower-bound) & 0.00 $\pm$ 0.00 & 0.01 $\pm$ 0.00 & 0.10 $\pm$ 0.00 & 0.00 $\pm$ 0.00 & 0.01 $\pm$ 0.00 & 0.10 $\pm$ 0.00\\
 \cmidrule{2-8}
 & \textsc{Optimistic} & 0.10 $\pm$ 0.00 & 0.13 $\pm$ 0.00 & 0.32 $\pm$ 0.00 & 0.01 $\pm$ 0.00 & 0.15 $\pm$ 0.00 & 0.31 $\pm$ 0.00\\
 & \textsc{Virtual} & 0.09 $\pm$ 0.00 & 0.14 $\pm$ 0.00 & 0.32 $\pm$ 0.00 & 0.02 $\pm$ 0.00 & 0.16 $\pm$ 0.00 & 0.32 $\pm$ 0.00\\
 & \textsc{Single-Ref} & 0.12 $\pm$ 0.00 & 0.23 $\pm$ 0.00 & 0.27 $\pm$ 0.00 & 0.01 $\pm$ 0.00 & 0.25 $\pm$ 0.00 & 0.27 $\pm$ 0.00\\
 & \algoname & 0.13 $\pm$ 0.00 & 0.21 $\pm$ 0.00 & 0.48 $\pm$ 0.00 & 0.02 $\pm$ 0.00 & 0.25 $\pm$ 0.00 & 0.47 $\pm$ 0.00\\
 \bottomrule
\end{tabular}\end{center} 
\end{small}
\end{table*}

\begin{table*}[ht]
\begin{small}
\caption{Knapsack ratio on robust models using FGSM and PGD attacker and various online algorithms.}
\label{appendix:knap_ratio_robust}
 \begin{center}\begin{tabular}{ c c c c c c c c }
 \toprule
 & & \multicolumn{3}{c}{MNIST (knapscak ratio in \%)} & \multicolumn{3}{c}{CIFAR-10 (knapsack ratio in \%)}\\
 & Algorithm & $k=10$ & $k=100$ & $k=1000$ & $k=10$ & $k=100$ & $k=1000$ \\
 \midrule
 \multirow{6}{*}{FGSM}
  & \textsc{Naive} (lower bound) & 1.2 $\pm$ 0.1 & 2.2 $\pm$ 0.0 & 10.5 $\pm$ 0.1 & 9.9 $\pm$ 0.2 & 12.5 $\pm$ 0.1 & 19.7 $\pm$ 0.0\\
 \cmidrule{2-8}
 & \textsc{Optimistic} & 38.0 $\pm$ 0.5 & 26.5 $\pm$ 0.1 & 44.6 $\pm$ 0.1 & 48.8 $\pm$ 0.5 & 45.2 $\pm$ 0.1 & 48.3 $\pm$ 0.0\\
 & \textsc{Virtual} & 35.6 $\pm$ 0.4 & 27.0 $\pm$ 0.1 & 38.0 $\pm$ 0.1 & 50.9 $\pm$ 0.3 & 52.5 $\pm$ 0.1 & 50.0 $\pm$ 0.0\\
 &\textsc{Single-Ref} & 46.9 $\pm$ 0.5 & 45.2 $\pm$ 0.2 & 46.4 $\pm$ 0.1 & 59.7 $\pm$ 0.6 & 73.1 $\pm$ 0.3 & 41.3 $\pm$ 0.1\\
 & \algoname & 49.2 $\pm$ 0.4 & 41.2 $\pm$ 0.1 & 58.6 $\pm$ 0.1 & 66.2 $\pm$ 0.4 & 74.5 $\pm$ 0.1 & 70.5 $\pm$ 0.0\\
 \midrule
 \multirow{6}{*}{PGD}
 & \textsc{Naive} (lower bound) & 1.3 $\pm$ 0.1 & 2.4 $\pm$ 0.0 & 10.7 $\pm$ 0.1 & 11.9 $\pm$ 0.6 & 14.6 $\pm$ 0.2 & 21.8 $\pm$ 0.1\\
 \cmidrule{2-8}
 & \textsc{Optimistic} & 31.1 $\pm$ 0.5 & 24.4 $\pm$ 0.1 & 42.7 $\pm$ 0.1 & 46.0 $\pm$ 1.4 & 45.3 $\pm$ 0.3 & 49.2 $\pm$ 0.1\\
 & \textsc{Virtual} & 29.9 $\pm$ 0.4 & 26.3 $\pm$ 0.1 & 37.9 $\pm$ 0.1 & 49.4 $\pm$ 1.2 & 52.4 $\pm$ 0.3 & 51.6 $\pm$ 0.1\\
 &\textsc{Single-Ref} & 39.5 $\pm$ 0.5 & 41.8 $\pm$ 0.2 & 43.3 $\pm$ 0.1 & 56.1 $\pm$ 2.1 & 69.5 $\pm$ 0.9 & 42.0 $\pm$ 0.4\\
 & \algoname & 41.3 $\pm$ 0.4 & 39.7 $\pm$ 0.1 & 57.9 $\pm$ 0.1 & 63.4 $\pm$ 1.2 & 72.7 $\pm$ 0.3 & 71.2 $\pm$ 0.1\\
 \bottomrule
\end{tabular}\end{center} 
\end{small}
\end{table*}

\clearpage
\subsection{Additional results}

\paragraph{Same architecture} In addition to table~\ref{table:non_robust_table1} we also provide some results on MNIST where $f_s$ and $f_t$ always have the same architecture but have different weights. This is a slightly less challenging setting as shown in \citet{bose2020adversarial}, we also observe that in this setting the adversaries are very effective against the target model.

\begin{table*}[ht]
\begin{small}
\caption{Fool rate on non-robust models, where $f_s$ and $f_t$ have the same architecture, using FGSM and PGD attacker and various online algorithms.}
\label{appendix:comp_ratio_same_arch}
 \begin{center}\begin{tabular}{ c c c c c }
 \toprule
 & & \multicolumn{3}{c}{MNIST (Fool rate in \%)}\\
 & Algorithm & $k=10$ & $k=100$ & $k=1000$ \\
 \midrule
 \multirow{6}{*}{FGSM}
  & \textsc{Naive} (lower bound) & 73.5 $\pm$ 0.5 & 72.3 $\pm$ 0.4 & 72.6 $\pm$ 0.4\\
  & \textsc{Opt} (Upper-bound) & 100.0 $\pm$ 0.0 & 99.7 $\pm$ 0.0 & 98.6 $\pm$ 0.1\\
 \cmidrule{2-5}
 & \textsc{Optimistic} & 89.8 $\pm$ 0.4 & 86.0 $\pm$ 0.2 & 84.9 $\pm$ 0.2\\
 & \textsc{Virtual} & 90.3 $\pm$ 0.3 & 90.0 $\pm$ 0.2 & 88.1 $\pm$ 0.2\\
 &\textsc{Single-Ref} & 94.0 $\pm$ 0.3 & 96.3 $\pm$ 0.2 & 79.3 $\pm$ 0.3\\
 & \algoname & 96.9 $\pm$ 0.2 & 98.6 $\pm$ 0.1 & 97.5 $\pm$ 0.1\\
 \midrule
 \multirow{6}{*}{PGD}
 & \textsc{Naive} (lower bound) & 91.1 $\pm$ 0.5 & 90.2 $\pm$ 0.4 & 90.0 $\pm$ 0.3\\
 & \textsc{Opt} (Upper-bound) & 98.5 $\pm$ 0.2 & 98.0 $\pm$ 0.1 & 97.4 $\pm$ 0.1\\
 \cmidrule{2-5}
 & \textsc{Optimistic} & 95.3 $\pm$ 0.3 & 93.8 $\pm$ 0.2 & 93.5 $\pm$ 0.2\\
 & \textsc{Virtual} & 95.5 $\pm$ 0.3 & 95.2 $\pm$ 0.2 & 94.4 $\pm$ 0.2\\
 &\textsc{Single-Ref} & 96.7 $\pm$ 0.3 & 96.9 $\pm$ 0.2 & 92.0 $\pm$ 0.3\\
 & \algoname & 97.1 $\pm$ 0.3 & 97.6 $\pm$ 0.1 & 97.0 $\pm$ 0.1\\
 \bottomrule
\end{tabular}\end{center} 
\end{small}
\end{table*}


\clearpage

\section{Distribution of Values Observed By Online Algorithms}
\label{appendix:visualization_of_values_observed} 
In this section we further investigate performance disparity of online algorithms against robust and non-robust models for CIFAR-10 as observed in Tables \ref{table:non_robust_table1} and \ref{table:madry_challenge}. We hypothesize that one possible explanation can be found through analyzing the ratio distribution of values $\mathcal{V}_i$'s for unsuccessful and successful attacks as observed by the online algorithm when attacking each model type. However, note that eventhough an online adversary may employ a fixed attack strategy to craft an attack $x'=\textsc{ATT}(x)$ the scale of values in each setting are not strictly comparable as the attack is performed on different model types.
In other words, given an $\textsc{ATT}$ it is significantly more difficult to attack a robust model and thus we can expect a lower $\mathcal{V}_i$ when compared to attacking a non-robust model. Thus to investigate the difference in efficacy of online attacks we pursue a distributional argument. %
%More precisely, we focus on the ratio distribution of values for unsuccessful and successful attacks of the entire data stream. 

Indeed, distributions of $\mathcal{V}_i$'s observed, for a specific permutation of $\mathcal{D}$, may drastically affect the performance of the online algorithms. Consider for instance, if the $\mathcal{V}_i$'s that correspond to successful attacks cannot be distinguished from the ones that are unsuccessful. In such a case one cannot hope to use an online algorithm---that only observes $\mathcal{V}_i$'s---to always correctly pick successful attacks. In Figure \ref{fig:distrib_values} we visualize the distribution of $\mathcal{V}_i$'s of unsuccessful and successful attacks as a density ratio for CIFAR-10 robust and non-robust models. We plot the distribution of $\mathcal{V}_i$'s (x-axis) with respect to the density ratio of unsuccessful versus successful attack vectors (y-axis) as provided by a kernel density estimator. As observed, the tail of the distribution in the non-robust case is heavier than in the robust case which indicates that there are many data points with high values that lead to unsuccessful attacks. Furthermore, this also suggests one explanation for the higher efficacy of online algorithms against robust models: fewer attacks are successful but they are easier to differentiate from unsuccessful ones. Importantly, this implies that given an online attack budget $k \ll n$ higher online fool rates can be achieved against robust models as the selected data points turn adversarial with higher probability when compared to non-robust models. 


%As observed, the tail of the distribution for unsuccessful attacks is heavier in the non-robust case than in the robust case indicating even when $\mathcal{A}$ observes high values a fraction of selected data points still lead to unsuccessful attack vectors. 

\begin{figure}[H]
\centering
    \includegraphics[width=\textwidth]{Figures/density_ratio.pdf}
    \caption{Distribution of the values for robust and non-robust models. We use a gaussian kernel density estimator to estimate the density.}
\label{fig:distrib_values}
\end{figure}

